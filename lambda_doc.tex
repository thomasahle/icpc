\documentclass[a4paper,final,8pt]{article}

\usepackage[landscape,twocolumn,top=1in, bottom=.5in, left=.8in, right=.8in]{geometry}

% Language and encoding
\usepackage[utf8]{inputenc}
\usepackage[T1]{fontenc}
\usepackage[english]{babel}
%\usepackage[protrusion=true,expansion=true]{microtype} % better typography

% Math
\usepackage{amsmath,amssymb,amsfonts,amsthm}
\usepackage{mathabx}

\newtheorem{thm}{Theorem}[section]
\newtheorem{defn}{Definition}[section]

\newcommand{\bbN}{\mathbb N} %the natural numbers
\newcommand{\bbZ}{\mathbb Z} %the integers
\newcommand{\bbQ}{\mathbb Q} %the rational numbers
\newcommand{\bbR}{\mathbb R} %the real numbers
\newcommand{\bbC}{\mathbb C} %the complex numbers

% Graphic stuff
\usepackage[pdftex]{graphicx}
\usepackage[usenames,dvipsnames,table]{xcolor}
\definecolor{shade}{RGB}{245,245,245}
\usepackage[pdftex,colorlinks=true]{hyperref}
\hypersetup
{
    bookmarksnumbered,
    linkcolor=RoyalBlue,
    anchorcolor=RoyalBlue,
    citecolor=RoyalBlue,
    urlcolor=RoyalBlue,
    pdfstartview={FitV},
    pdfdisplaydoctitle
}

% Tables
\usepackage{booktabs}
\usepackage[hang,small,bf]{caption}

% Debug, etc.
\usepackage{todonotes}

\pagestyle{myheadings}
\markright{University of Copenhagen}

% Computer science stuff
\usepackage{verbatim}
\newcommand{\mono}[1]{{\ttfamily#1}}
\usepackage{listings}
\lstset
{
    tabsize=2,
    numbers=left,
    breaklines=true,
    backgroundcolor=\color{shade},
    framexleftmargin=0.05in,
    basicstyle=\ttfamily\small,
    numberstyle=\tiny,
    keywordstyle=\color{RoyalBlue},
    stringstyle=\color{Maroon},
    commentstyle=\color{ForestGreen},
    language=C++
}


\title{Team Reference \\ University of Copenhagen}
\author{Lambdabamserne}

\begin{document}

\maketitle

\tableofcontents

\section{Fenwick Tree}

\textbf{Note:} The arrays are 1-indexed!

\begin{lstlisting}
const int MAX_N = 200;
int N;

int phi(int n){ return n & -n; }

int arr[MAX_N+1]; //arr[k] = x[k-phi(k)+1] + ... + x[k]

int get(int k) { //returns x[1] + x[2] + x[3] + .. + x[k]
    return k <= 0 ? 0 : arr[k] + get(k-phi(k));
}

void add(int k, int c) { // x[k] += c
    if(k > N) return;
    arr[k] += c;
    add(k+phi(k), c);
}
\end{lstlisting}

With loops instead:

\begin{lstlisting}
const int MAX_N = 200;
int N;

int phi(int n){ return n & -n; }

int arr[MAX_N+1]; //arr[k] = x[k-phi(k)+1] + ... + x[k]

int get(int k) { //returns x[1] + x[2] + x[3] + .. + x[k]
    int sum = 0;
    for(int i = k;i > 0;i -= phi(i)) sum += arr[i];
    return sum;
}

void add(int k, int c) { // x[k] += c
    for(int i = k;i <= N;i += phi(i)) arr[i] += c;
}
\end{lstlisting}


Two dimensional:

\begin{lstlisting}
const int MAX_N = 50;
int N;

int phi(int n){ return n & -n; }

//arr[k][l] = sum x[i][j] hvor k-phi(k) < i <= k, l-phi(l) < j <= l
int arr[MAX_N+1][MAX_N+1];

int get(int k, int l) { //returns sum x[i][j], hvor i <= k,j <= l
    int sum = 0;
    for(int i = k;i > 0;i -= phi(i))
    for(int j = l;j > 0;j -= phi(j))
    sum += arr[i][j];
    return sum;
}

void add(int k, int l, int c) { // x[k] += c
    for(int i = k;i <= N;i += phi(i))
    for(int j = l;j <= N;j += phi(j))
    arr[i][j] += c;
}
\end{lstlisting}


\section{Treap}

Following is example treap code solving the movie task from nwerc 2011 (iirc).

% MAKE SURE treap.cpp IS WORKING CODE OR WE WILL BE FUCKED AT THE CONTEST!
\lstinputlisting{src/treap.cpp}

\section{Hungarian Algorithm}

Hungarian algorithm solution for SPOJ 286. Should be easy to fit to another
problem. More info in comments of code.

\lstinputlisting{src/hungarian.cpp}

\section{Suffix Array}

\begin{lstlisting}
// Begins Suffix Arrays implementation
// O(n log n) - Manber and Myers algorithm
// Refer to "Suffix arrays: A new method for on-line string searches",
// by Udi Manber and Gene Myers

//Usage:
// Fill str with the characters of the string.
// Call SuffixSort(n), where n is the length of the string stored in str.
// That's it!

//Output:
// pos = The suffix array. Contains the n suffixes of str sorted in lexicographical order.
// Each suffix is represented as a single integer (the position of str where it starts).
// rank = The inverse of the suffix array. rank[i] = the index of the suffix str[i..n)
//      in the pos array. (In other words, pos[i] = k <==> rank[k] = i)
//      With this array, you can compare two suffixes in O(1): Suffix str[i..n) is smaller
//      than str[j..n) if and only if rank[i] < rank[j]

int str[N]; //input
int rank[N], pos[N]; //output
int cnt[N], next[N]; //internal
bool bh[N], b2h[N];

// Compares two suffixes according to their first characters
bool smaller_first_char(int a, int b){
    return str[a] < str[b];
}

void suffixSort(int n){
    //sort suffixes according to their first characters
    for (int i=0; i<n; ++i){
        pos[i] = i;
    }
    sort(pos, pos + n, smaller_first_char);
    //{pos contains the list of suffixes sorted by their first character}

    for (int i=0; i<n; ++i){
        bh[i] = i == 0 || str[pos[i]] != str[pos[i-1]];
        b2h[i] = false;
    }

    for (int h = 1; h < n; h <<= 1){
    //{bh[i] == false if the first h characters of pos[i-1] == the first h characters of pos[i]}
        int buckets = 0;
        for (int i=0, j; i < n; i = j){
            j = i + 1;
            while (j < n && !bh[j]) j++;
            next[i] = j;
            buckets++;
        }
        if (buckets == n) break; // We are done! Lucky bastards!
        //{suffixes are separted in buckets containing strings starting with the same h characters}

        for (int i = 0; i < n; i = next[i]){
            cnt[i] = 0;
            for (int j = i; j < next[i]; ++j){
                rank[pos[j]] = i;
            }
        }

        cnt[rank[n - h]]++;
        b2h[rank[n - h]] = true;
        for (int i = 0; i < n; i = next[i]){
            for (int j = i; j < next[i]; ++j){
                int s = pos[j] - h;
                if (s >= 0){
                    int head = rank[s];
                    rank[s] = head + cnt[head]++;
                    b2h[rank[s]] = true;
                }
            }
            for (int j = i; j < next[i]; ++j){
                int s = pos[j] - h;
                if (s >= 0 && b2h[rank[s]]){
                    for (int k = rank[s]+1; !bh[k] && b2h[k]; k++) b2h[k] = false;
                }
            }
        }
        for (int i=0; i<n; ++i){
            pos[rank[i]] = i;
            bh[i] |= b2h[i];
        }
    }
    for (int i=0; i<n; ++i){
        rank[pos[i]] = i;
    }
}
// End of suffix array algorithm

// Begin of the O(n) longest common prefix algorithm
// Refer to "Linear-Time Longest-Common-Prefix Computation in Suffix
// Arrays and Its Applications" by Toru Kasai, Gunho Lee, Hiroki
// Arimura, Setsuo Arikawa, and Kunsoo Park.
int height[N];
// height[i] = length of the longest common prefix of suffix pos[i] and suffix pos[i-1]
// height[0] = 0
void getHeight(int n){
    for (int i=0; i<n; ++i) rank[pos[i]] = i;
    height[0] = 0;
    for (int i=0, h=0; i<n; ++i){
        if (rank[i] > 0){
            int j = pos[rank[i]-1];
            while (i + h < n && j + h < n && str[i+h] == str[j+h]) h++;
            height[rank[i]] = h;
            if (h > 0) h--;
        }
    }
}
// End of longest common prefixes algorithm
\end{lstlisting}


\section{Z algorithm}

Given a string $S$, produces an array $Z$, where $Z[i]$ is the length of the
longest substring starting at $S[i]$, that is also a prefix of $S$.

\begin{lstlisting}
int L = 0, R = 0;
for (int i = 1; i < n; i++) {
    if (i > R) {
        L = R = i;
        while (R < n && s[R-L] == s[R]) R++;
        z[i] = R-L; R--;
    } else {
        int k = i-L;
        if (z[k] < R-i+1) z[i] = z[k];
        else {
            L = i;
            while (R < n && s[R-L] == s[R]) R++;
            z[i] = R-L; R--;
        }
    }
}
\end{lstlisting}


\section{Maximum Flow}

\lstinputlisting{src/dinic.cpp}

\section{Minimum cost flow}

Min Cost flow solution for codeforces 277E.

\lstinputlisting{src/mincost.cpp}

\section{Bridge Finding}

\begin{lstlisting}
bool M[128][128]; // adjacency matrix
int colour[128]; // 0 is white, 1 is gray and 2 is black
int dfsNum[128], num; // DFS numbers
int n; // the number of vertices
// returns the smallest DFS number that has a back edge pointing to it
// in the DFS subtree rooted at u
int dfs( int u, int p ) {
    colour[u] = 1;
    dfsNum[u] = num++;
    int leastAncestor = num;
    for( int v = 0; v < n; v++ ) if( M[u][v] && v != p ) {
        if( colour[v] == 0 ) {
            // (u,v) is a forward edge
            int rec = dfs( v, u );
            if( rec > dfsNum[u] )
                cout << "Bridge: " << u << " " << v << endl;
            leastAncestor = min( leastAncestor, rec );
        } else {
            // (u,v) is a back edge
            leastAncestor = min( leastAncestor, dfsNum[v] );
        }
    }
    colour[u] = 2;
    return leastAncestor;
}

int main() {
    // ... fill up M[][] with an adjacency matrix
    // ... set n to be the number of vertices
    for( int i = 0; i < n; i++ ) colour[i] = 0;
    num = 0;
    dfs( 0, -1 );
}
\end{lstlisting}


\section{Sweepline}

I believe this is a sweep line that finds the two closest points in the plane.

\begin{lstlisting}
#include <stdio.h>
#include <set>
#include <algorithm>
#include <cmath>

using namespace std;

#define px second
#define py first

typedef pair<long long, long long> pairll;
int n;
pairll pnts [100000];
set<pairll> box;
double best;

int compx(pairll a, pairll b) { return a.px<b.px; }

int main () {
    scanf("%d", &n);
    for (int i=0;i<n;++i)
        scanf("%lld %lld", &pnts[i].px, &pnts[i].py);

    sort(pnts, pnts+n, compx);
    best = 1500000000; // INF
    box.insert(pnts[0]);
    int left = 0;

    for (int i=1;i<n;++i)
    {
        while (left<i && pnts[i].px-pnts[left].px > best)
            box.erase(pnts[left++]);

        for (set<pairll>::iterator it=box.lower_bound(make_pair(pnts[i].py-best, pnts[i].px-best));
                it !=box.end() && pnts[i].py+best>=it->py;
                it++)
            best = min(best, sqrt(pow(pnts[i].py - it->py, 2.0)+pow(pnts[i].px - it->px, 2.0)));
        box.insert(pnts[i]);
    }
    printf("%.2f\n", best);
    return 0;
}
\end{lstlisting}


\section{Topological Sorting}

I have not tested this!

\lstinputlisting{src/top_sort.cpp}


\section{Minimum Spanning Tree}

Prim's algorithm: $O(m\lg n)$. This uses a set instead of a priority queue,
so it is $O(n)$ space, but probably a little slower. Don't think it matters
too much. I don't think it works with negative edges.

\lstinputlisting{src/prim_mst.cpp}


\section{LCA}

This is the very simple $\langle N\lg N, \lg N\rangle$ solution, so each
query is $O(\lg N)$, which should be enough for most problems.
I have not tested it very thoroughly.

\lstinputlisting{src/log_lca.cpp}


\section{RMQ}
Lav det med et segment tree eller et treap eller lignende. Det er nemmest.
Ellers kan man gøre det i $O(1)$ med LCA, men har ikke $O(1)$ LCA algoritmen
her, da jeg ikke kunne finde en god implementation.


\section{Floyd-Warshall}

\begin{lstlisting}
for (int k = 0; k < n; ++k)
    for (int i = 0; i < n; ++i)
        for (int j = 0; j < n; ++j)
            dist[i][j] = min(dist[i][j], dist[i][k] + dist[k][j]);
\end{lstlisting}

\section{2SAT}

\lstinputlisting{src/2sat.cpp}

\section{Tarjan's strongly-connected-components}

\lstinputlisting{src/tarjan.cpp}

\section{Geometry}

\begin{lstlisting}
#include <iostream>
#include <cstdio>
#include <complex>
typedef std::complex<double> com;

using namespace std;

com circumcenter(com a, com b, com c) {
    return (a * conj(a) * (c-b) + b * conj(b) * (a-c) + c * conj(c) * (b-a))/
           ( conj(a) * (c-b) + conj(b) * (a-c) + conj(c) * (b-a));
}
com centroid(com a, com b, com c) {
    return (a+b+c)/3.0;
}
com orthocenter(com a, com b, com c) {
    return a+b+c - 2.0 * circumcenter(a,b,c); //2O + H = 3G
}
double area(com a, com b, com c) {
    return abs(0.5 * imag((a-b)*conj(c-b)));
}
double turn(com a, com b, com c) {
    return imag((a-b)*conj(c-b));
}
int turn2(com a, com b, com c) {
    double x = imag((a-b)*conj(c-b));
    return x < 0 ? -1 : x > 0;
}
double area(com p[], int n) {
    double sum = 0;
    for(int i=1;i+1<n;i++) {
        sum += turn(p[0],p[i],p[i+1]); //imag((p[i+1]-p[0])*conj(p[i]-p[0]))
    }
    return abs(0.5 * sum);
}
com omega(com x, com y) {
    return conj(x-y)/(x-y);
}
com rho(com x, com y) {
    return (conj(x) * y - x * conj(y))/(x-y);
}
com intersection(com a, com b, com c, com d) {
    return (rho(a,b) - rho(c,d))/(omega(a,b) - omega(c,d));
}
bool straddles(com a, com b, com c, com d) {
    return turn2(c, d, a) * turn2(c, d, b) == -1 &&
           turn2(a, b, c) * turn2(a, b, d) == -1;
}
bool straddlesInclEndpoints(com a, com b, com c, com d) {
    return turn2(c, d, a) * turn2(c, d, b) <= 0 &&
           turn2(a, b, c) * turn2(a, b, d) <= 0;
}
com proj(com a, com b, com p) {
    return 0.5 * conj(p * omega(a,b) + conj(p) - rho(a,b));
}
bool hasLineCirleIntersection(com a, com b, com c, double r) {
    com p = proj(a, b, c);
    double dist = abs(p-c);
    return dist <= r;
}
pair<com, com> lineCircleIntersection(com a, com b, com c, double r) {
    com p = proj(a, b, c);
    double dist = abs(p-c);
    if(dist > r) return make_pair(0, 0);
    com ex = sqrt(pow(r/dist, 2) - 1) * (p-c) * com(0,1);
    return make_pair( p + ex, p - ex);
}

com radicalAxis(com c1, double r1, com c2, double r2) {
    double d = abs(c1-c2);
    return 0.5 * (1 + (r1*r1 - r2*r2)/(d*d)) * (c2-c1) + c1;
}
bool hasCircleCircleIntersection(com c1, double r1, com c2, double r2) {
    return abs(r1 - r2) <= abs(c1-c2) && abs(c1-c2) <= r1+r2;
}
pair<com, com> circleCircleIntersection(com c1, double r1, com c2, double r2) {
    com p = radicalAxis(c1, r1, c2, r2);
    double dist = abs(p-c1);
    if(dist > r1) return make_pair(0, 0);
    com ex = sqrt(pow(r1/dist, 2) - 1) * (p-c1) * com(0,1);
    return make_pair(p + ex, p - ex);
}

bool lexCmp(com a, com b) {
    return a.real() < b.real() || (a.real() == b.real() && a.imag() < b.imag());
}
void convexHull (vector<com>& a) {
    if (a.size () == 1) return;
    sort(a.begin(), a.end(), lexCmp);
    com p1 = a[0], p2 = a.back();
    vector<com> up, down;
    up.push_back(p1);
    down.push_back(p1);
    for (size_t i = 1; i < a.size(); ++i) {
        if(i == a.size()-1 || turn(p1, a[i], p2) < 0) {
            while(up.size() >= 2 && turn(up[up.size()-2], up[up.size ()-1], a[i]) >= 0) {
                up.pop_back();
            }
            up.push_back(a[i]);
        }
        if (i == a.size() -1 || turn(p1, a[i], p2) > 0) {
            while (down.size() >= 2 && turn(down[down.size () -2], down[down.size () -1], a[i]) <= 0)
                down.pop_back ();
            down.push_back (a [i]);
        }
    }
    a.clear();
    for (size_t i = 0; i <up.size(); ++i)
        a.push_back(up[i]);
    for (size_t i = down.size()-2; i > 0; --i)
        a.push_back(down[i]);
}
\end{lstlisting}

\section{Union-Find}

\lstinputlisting{src/dsu.cpp}

\section{Segment Tree}
Mathias' kode. Han ved hvordan den fungerer.

\lstinputlisting{src/segtree.cpp}

\section{Knuth-Morris-Pratt}

Stjålet fra Stanford.

\lstinputlisting{src/kmp.cpp}

\section{Diverse talteori}

\lstinputlisting{src/number_theory.cpp}

\section{Gauss Jordan}

\lstinputlisting{src/gauss_jordan.cpp}

\section{Delaunay}

\lstinputlisting{src/delaunay.cpp}

\section{Maximum bipartite matching}

\lstinputlisting{src/max_match.cpp}

\section{Minimum cost matching}

\lstinputlisting{src/min_cost_match.cpp}

\section{Dijkstra template}

\begin{lstlisting}
#define NUMV 1 // the maximum number of vertices
#define INF ~(1 << 31)

struct Node {
    Node (int index_, int length_) : index(index_), length(length_) {}
    int index, length;
    inline bool operator<(const Node& o) const {
        return length > o.length;
    }
};

int best[NUMV];
vector<Node> adj[NUMV];

// best[i] = keeps track of the best(shortest) distance from "start" to "i"
void djikstra(int start) {
    memset(best, INF, sizeof(int)*NUMV); // initialize to infinity
    best[start] = 0;
    priority_queue<Node> pq;
    pq.push(Node(start, 0));
    while (!pq.empty()) {
        Node curr = pq.top();
        pq.pop();
        if (curr.length > best[curr.index]) // if it's worse than what you already have, continue
            continue;

        // go throught the adjacency list
        for (vector<Node>::iterator it = adj[curr.index].begin(); it != adj[curr.index].end(); it++) {
            if (curr.length + it->length < best[it->index]) { // update if necessary
                best[it->index] = curr.length + it->length;
                pq.push(Node(it->index, best[it->index]));
            }
        }
    }
}
\end{lstlisting}

To add an edge $(v,u)$ you should call \texttt{adj[v].push\_back(u)} and
\texttt{adj[u].push\_back(v)}, where \texttt{u} and \texttt{v} are the ids of
those nodes.

\section{Graph Theorems}

\begin{itemize}
\item En graf med $\Omega(n^{1 + 1/k})$ kanter maa indeholde en kreds af laengde hoejst $2k$, for ethvert positivt heltal $k$.

\item En graf er bipartit hvis og kun hvis alle dens kredse er af lige laengde.

\item Size(min vertex cover) = Size(max matching) i en bipartit graf.

\item I enhver graf $G$ findes en bipartit delgraf indeholdende mindst halvdelen af kanterne fra $G$.

\item En graf med $n$ knuder, der ikke indeholder en clique af størrelse $r+1$ kan højest indeholde $((r-1)/r)(n^2/2)$ kanter.

\item Antallet af knuder med ulige valens er lige.

\item Every graph of order $n\ge 6$ and size at least $3n - 5$ contains two disjoint cycles.

\item Every graph of order $n\ge 5$ and size at least $n + 4$ contains two edge-disjoint cycles.

\item A planar graph cannot contain $K_5$ or $K_{3,3}$.
\end{itemize}

\section{Diverse theorems}
On a simple polygon constructed on a grid of equal-distanced points, for area
$A$, number of interior points $I$, number of boundary points $B$, we have
$A=I+B/2-1$.

\section{Summaries of selected algorithms}
\begin{description}
    \item [Fenwick Tree] Can be used to increase/decrease specific array
        indices and sum concecutive cells in $O(\log n)$.
    \item [Z algorithm] String algorithm, which can be used for various things.
    \item [Topological sort] Sorts vertices of a directed acyclic graph such
        that for each edge $(v,u)$ we have $v$ before $u$ in the ordering.
    \item [LCA] Given a tree $T$, computes a data structure such that we can
        find the nearest common ancestor of any two nodes in $O(\log n)$ time.
    \item [Geometry] A point $(x,y)$ is represented by a complex number $x +
        yi$, and is instantiated by \texttt{com(x,y)}.
    \item [Talteori] Misc useful functions such as \texttt{gcd} and
        \texttt{lcm}.
    \item [Maximum bipartite matching] Finds the maximum matching in a
        bipartite graph. Useful in problems where things has to be assigned
        with unit cost.
\end{description}

\end{document}
